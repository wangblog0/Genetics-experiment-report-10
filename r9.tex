% !TEX program = xelatex
% -*- coding: utf-8 -*-
\documentclass[a4paper]{article}

% -------------------- 中文支持 (XeLaTeX) --------------------
% 推荐使用 XeLaTeX 编译以获得更好的中文字体支持
\usepackage{fontspec}            % XeLaTeX 字体接口
\usepackage{xeCJK}               % 中文支持
\setCJKmainfont[AutoFakeBold=true]{SimSun} % 正文中文(可替换为 SimHei, 宋体, 思源宋体等)
\setCJKsansfont[AutoFakeBold=true]{SimHei}         % 无衬线中文(黑体)
\setCJKmonofont{FangSong}       % 等宽中文(可选)
\setmainfont{Times New Roman}    % 英文字体
\setsansfont{Arial}

% -------------------- 编码与语言 --------------------
\usepackage{babel}
\usepackage{microtype}           % 文本微调(对 XeLaTeX 有部分支持)

% -------------------- 页面布局 --------------------
\usepackage[a4paper,margin=2.5cm]{geometry}
\usepackage{setspace}
\onehalfspacing

% -------------------- 图表与浮动体 --------------------
\usepackage{graphicx}
\usepackage{float}
\usepackage{subcaption}
\usepackage{booktabs}
\usepackage{caption}
\usepackage{tikz}

\captionsetup{font=small,labelfont=bf}

% -------------------- 数学与代码 --------------------
\usepackage{amsmath,amssymb,amsthm}
\usepackage{siunitx}             % 物理量与单位
\usepackage{listings}            % 源代码显示(如需要)
\lstset{
    basicstyle=\ttfamily\small,
    breaklines=true,
    breakatwhitespace=false,
    breakindent=0pt,
    breakautoindent=false,
    columns=flexible,
    keepspaces=true,
    literate={A}{A\allowbreak}1 {T}{T\allowbreak}1 {C}{C\allowbreak}1 {G}{G\allowbreak}1
}

% -------------------- 超链接 --------------------
\usepackage[colorlinks=true,linkcolor=blue,citecolor=blue,urlcolor=blue]{hyperref}

% -------------------- 页眉页脚 --------------------
\usepackage{fancyhdr}
\pagestyle{fancy}
\fancyhf{}
\fancyhead[L]{mtDNA进化分析}
\fancyhead[R]{王翔宇 2023521039}
\fancyfoot[C]{\thepage}

% 使得 document 中使用 plain 页眉样式的页面(如 \maketitle)也显示相同的页眉页脚
\fancypagestyle{plain}{%
    \fancyhf{}%
    \fancyhead[L]{mtDNA进化分析}%
    \fancyfoot[C]{\thepage}%
}

% -------------------- 其他实用宏包 --------------------
\usepackage{enumitem}            % 自定义列表样式
\usepackage{longtable}           % 跨页表格
\usepackage{multirow}
\usepackage{xcolor}
\usepackage{indentfirst}
\usepackage{cleveref}
\usepackage{dnaseq}
\usepackage{titlesec}            % 标题格式定制

% -------------------- 标题格式设置 --------------------
% 设置section标题为黑体加粗
\titleformat{\section}
    {\Large\sffamily\bfseries}  % sffamily调用黑体,bfseries加粗
    {\thesection}
    {1em}
    {}

% 设置subsection标题为黑体加粗
\titleformat{\subsection}
    {\large\sffamily\bfseries}
    {\thesubsection}
    {1em}
    {}

% 设置subsubsection标题为黑体加粗
\titleformat{\subsubsection}
    {\normalsize\sffamily\bfseries}
    {\thesubsubsection}
    {1em}
    {}

% -------------------- 文档开始 --------------------
\begin{document}

\title{\sffamily\textbf{mtDNA进化分析}}
\author{王翔宇 2023521039}
\date{}
\maketitle

% 后续正文占位(用户可在这里继续写各部分)
% 下面我会在 TODO 中添加一个任务去插入完整正文结构

\section{实验名称}
\noindent mtDNA进化分析

\section{实验目的}
\begin{enumerate}
    \item 了解线粒体DNA (mtDNA) 的结构与功能
    \item 掌握口腔上皮细胞基因组DNA的提取方法
    \item 学习用软件对获得的DNA序列进行初步分析的方法
    \item 对序列比对结果进行单核苷酸多态性的分析,从而追踪人类进化的轨迹,并进行亲缘关系的分析
\end{enumerate}

\section{实验原理}

\subsection{线粒体的基本结构与功能}
线粒体是细胞中的重要细胞器,其结构与功能直接关联到mtDNA的遗传特性:
\begin{itemize}
    \item \textbf{结构特征}:
    \begin{itemize}
        \item 光镜下呈颗粒状,电镜下可见双层膜结构及嵴(内膜折叠形成),这为氧化磷酸化提供了巨大表面积。
        \item 作为“能量工厂”,线粒体分布有三羧酸循环系统和电子传递系统,是细胞产生ATP(腺苷三磷酸)的主要场所。
        \item 线粒体是核外唯一含有遗传效应物质的细胞器,这意味着mtDNA独立于核基因组,具有独特的遗传规律。
    \end{itemize}
    \item \textbf{功能角色}:mtDNA编码部分线粒体蛋白,参与能量代谢,其突变可能影响细胞功能,如与Leber遗传性视神经病等疾病相关。
\end{itemize}

\subsection{线粒体DNA的遗传学特征}
mtDNA的遗传方式具有特殊性,下列六个关键特征决定了其在进化分析和亲缘关系研究中的价值:
\begin{enumerate}
    \item \textbf{母系遗传}:mtDNA通常通过母系传递,但也有研究(如Lu等、Rius等)探讨父系遗传的可能性,结论是双亲遗传并非常见,临床实践中仍以母系遗传为主。
    \item \textbf{异质性}:个体内可能存在多种mtDNA序列变体,这源于突变或细胞分裂中的不均匀分布(即异质质/均质质现象)。
    \item \textbf{阈值效应}:mtDNA突变需积累到一定比例才会表现表型,这与线粒体的功能冗余相关。
    \item \textbf{半自主性}:mtDNA部分依赖核基因组进行复制和表达,但拥有独立的遗传系统。
    \item \textbf{高突变率}:mtDNA的突变率比核DNA高约10倍,原因包括线粒体缺乏高效的DNA修复机制且暴露于氧化应激环境。
    \item \textbf{多态性}:mtDNA在人群中呈现高度变异,为人类进化和疾病研究提供了分子标记。
\end{enumerate}

\subsection{mtDNA的多态性类型及其机制}
文档将mtDNA多态性分为两类,这直接关联到实验中的序列分析方法:
\begin{itemize}
    \item \textbf{点的多态性}:指单个碱基的突变(如碱基替换),可能导致限制性内切酶位点的丢失或新增,可通过Southern杂交或酶切图谱等技术检测。例如,Leber遗传性视神经病与nt11778 G$>$A突变相关,体现了点突变的功能影响。
    \item \textbf{序列多态性}:由缺失、重复或插入事件引起,或由于高变区内串联重复序列的拷贝数差异所致。这类多态性不一定改变酶切位点的碱基,但会改变其在基因组中的相对位置,常需通过PCR与测序来分析。序列多态性主要集中在D-loop(控制区),包括高变区HV1(nt16024--nt16365,342 bp)和HV2(nt73--nt340,268 bp),这些区域变异速率高,是实验的重点分析靶点。
\end{itemize}

\subsection{高变区(HV1和HV2)的关键作用}
高变区HV1和HV2是mtDNA中变异最集中的区域,碱基变异速率高,适用于追踪人类进化轨迹和进行群体分析。实验原理中常用于PCR扩增的引物示例包括HV1引物(如L15997和H16391),可用于扩增约430 bp的HV1片段;HV2引物用于扩增约400 bp的HV2片段。通过扩增并测序这些区域,结合参考序列(如Revised Cambridge Reference Sequence或RSRS)进行比对与SNP分析,可识别个体间的差异并初步判断mtDNA单倍群归属。

以上内容已插入实验原理节,便于后续对PCR、酶切及序列分析的原理与实验设计进行引用和展开。

\section{实验试剂和耗材}

\begin{enumerate}
    \item 材料:人口腔上皮细胞 
    \item 试剂:灭菌水,1 × TAE,Ezup 柱式口腔拭子基因组 DNA 抽提试剂盒,2 × Taq Master Mix, Marker,loading buffer, 引物 
    \item 仪器:PCR仪,高速离心机,金属浴,电泳仪 
    \item 其他用品:一次性纸杯,1.5 ml 离心管,0.2 ml PCR反应管,各种规格灭菌吸头,各量程移液器
\end{enumerate}



\section{实验步骤}

\subsection{样本采集与DNA提取}
\subsubsection*{实验前准备}
\begin{itemize}
    \item \textbf{自备材料}:小型高速离心机(最大离心力≥12,000×g)、水浴锅、2 ml 离心管、无水乙醇、RNase A 溶液(10 mg/ml)。
    \item \textbf{试剂盒预处理}:
    \begin{itemize}
        \item 检查 Buffer ACL 和 Buffer CL 是否有沉淀,如有则在 65℃ 溶解后使用。
        \item 在 Wash Solution 中加入规定量的无水乙醇(15 ml Wash Solution 加 45 ml 无水乙醇)。
    \end{itemize}
\end{itemize}

\subsubsection*{标准操作流程}
\begin{enumerate}
    \item \textbf{样本采集与预处理}
    \begin{itemize}
        \item 使用口腔拭子在口腔内壁擦拭 6--10 次,晾干 2 小时后保存。
        \item 注意事项:取样前 30 分钟内勿进食或饮水,避免样本污染。
    \end{itemize}

    \item \textbf{细胞裂解}
    \begin{enumerate}[label=\arabic*)]
        \item 将拭子棉签部分剪入 2 ml 离心管,加入 400 $\mu$l Buffer PBS。
        \item 加入 400 $\mu$l Buffer ACL、200 $\mu$l Buffer CL 和 20 $\mu$l Proteinase K。
        \item 震荡混匀后,置 65℃ 水浴 10 分钟,期间间歇混匀。
    \end{enumerate}

    \item \textbf{DNA 结合与纯化}
    \begin{enumerate}[label=\arabic*)]
        \item 加入 400 $\mu$l 无水乙醇,充分混匀。
        \item 将全部溶液转移至硅胶膜吸附柱(吸附柱容量为 750 $\mu$l,若体积过大需分两次过柱)。
        \item 10,000 rpm 离心 1 分钟,弃废液。
    \end{enumerate}

    \item \textbf{洗涤步骤}
    \begin{enumerate}[label=\arabic*)]
        \item 加入 500 $\mu$l Wash Solution,10,000 rpm 离心 1 分钟,弃废液。
        \item 重复洗涤一次,确保彻底去除杂质。
    \end{enumerate}

    \item \textbf{DNA 洗脱}
    \begin{enumerate}[label=\arabic*)]
        \item 12,000 rpm 离心 2 分钟(关键步骤:去除残留乙醇)。
        \item 将吸附柱转入新 1.5 ml 离心管,在膜中央加入 50 $\mu$l TE Buffer。
        \item 静置 3 分钟后,12,000 rpm 离心 2 分钟,收集洗脱液。
        \item 获得的 DNA 溶液可于 -20℃ 保存或立即使用。
    \end{enumerate}
\end{enumerate}

\subsubsection*{质量保证}
\begin{itemize}
    \item 预期 OD$_{260}$/OD$_{280}$ 比值为 1.7--1.9。
    \item 无需苯酚/氯仿等有毒试剂,符合安全标准。
\end{itemize}

\subsection{PCR 扩增(与原流程一致)}
\subsubsection*{反应体系配制(50 $\mu$l 体系)}
\begin{center}
\begin{tabular}{lcc}
\toprule
组分 & 实验组 & 对照组 \\
\midrule
模板 DNA & 5 $\mu$l & 5 $\mu$l ddH$_2$O \\
引物1(10 $\mu$M) & 2 $\mu$l & 2 $\mu$l \\
引物2(10 $\mu$M) & 2 $\mu$l & 2 $\mu$l \\
2$\times$ Taq Master Mix & 25 $\mu$l & 25 $\mu$l \\
ddH$_2$O & 16 $\mu$l & 16 $\mu$l \\
\bottomrule
\end{tabular}
\end{center}

\subsubsection*{扩增程序}
\begin{itemize}
    \item 预变性:94℃ 5 分钟。
    \item 35 个循环:94℃ 30 秒 → 55--60℃ 30 秒 → 72℃ 1 分钟。
    \item 最终延伸:72℃ 10 分钟。
\end{itemize}

\subsection{电泳验证与测序}
\begin{enumerate}
    \item \textbf{琼脂糖凝胶电泳}:取 10 $\mu$l PCR 产物进行电泳分析,预期条带大小:HV1 区约 430 bp,HV2 区约 400 bp。
    \item \textbf{样本送测}:有条带的样本移交测序分析。
\end{enumerate}

\subsection{序列分析与 SNP 鉴定}
\begin{enumerate}
    \item \textbf{软件分析}:使用 MEGA/PHYLIP 等软件比对 rCRS/RSRS 参考序列,识别 HV1/HV2 区单核苷酸多态性(SNP)。
    \item \textbf{进化追踪}:与 RSRS 序列对比以确定单倍群,并分析个体间差异位点用于进化或亲缘关系推断。
\end{enumerate}

\subsection{常见问题解决方案}
\begin{enumerate}
    \item \textbf{得率低}:检查拭子类型(推荐植绒拭子)、确认 Wash Solution 中乙醇添加量、检查洗脱液 pH(建议 >7.5)。
    \item \textbf{DNA 降解}:避免样本反复冻融,裂解时间不超过 30 分钟。
    \item \textbf{杂质残留}:离心后开盖风干或轻轻孵育以去除乙醇残留。
    \item \textbf{电泳异常}:建议使用 0.8\% 琼脂糖凝胶,适当减少上样量并检查 Marker/缓冲液浓度。
\end{enumerate}

\section{数据记录}
\subsection{电泳结果}
电泳结果如图~\ref{fig:electrophoresis} 所示,我的样品编号为6。(关于1号样品出现这种现象的解释:胶没凝好就拔孔导致第一个孔漏了,于是我们又在第八个空补了一点,但是已经跑了一会儿了,所以这条带比其他条带要靠后一点。)
\begin{figure}[htbp!]
    \centering
    \includegraphics[width=0.6\textwidth]{electrophoresis.png}
    \caption{电泳结果图}
    \label{fig:electrophoresis}
\end{figure}

\subsection{测序数据}
HV1区测序正常,可以直接从测序公司获取拼接后的序列数据。下图截取了HV1,F方向引物的部分测序结果(图~\ref{fig:hv1_seq}),可以看到信号较好,碱基识别清晰。
\begin{figure}[htbp!]
    \centering
    \includegraphics[width=0.8\textwidth]{hv1_seq.png}
    \caption{HV1区F方向引物测序结果截图}
    \label{fig:hv1_seq}
\end{figure}

HV2去测序结果不理想,公司分别提供了F和R两个方向的测序数据,F方向出现套峰,R方向被报告为中断,但使用SnapGene尝试拼接,仍获得一段可用序列。人工查看F方向和R测序结果,峰还比较清晰,可以继续分析。F和R方向的部分测序结果以及最后的拼接结果示意如图~\ref{fig:hv2_seq} 所示。
\begin{figure}[htbp!]
    \centering
    \begin{subfigure}[b]{0.49\textwidth}
        \centering
        \includegraphics[width=\textwidth]{hv2_f.png}
        \caption{HV2区F方向引物测序结果截图}
    \end{subfigure}
    \hfill
    \begin{subfigure}[b]{0.49\textwidth}
        \centering
        \includegraphics[width=\textwidth]{hv2_r.png}
        \caption{HV2区R方向引物测序结果截图}    
    \end{subfigure}
    \vfill
    \begin{subfigure}[b]{0.8\textwidth}
        \centering
        \includegraphics[width=\textwidth]{hv2_assemble.png}
        \caption{HV2区F和R方向引物测序结果拼接示意图}        
    \end{subfigure}
    \caption{HV2区测序结果}
    \label{fig:hv2_seq}
\end{figure}

HV1区测序结果如下:
\begin{lstlisting}
TATTCTCTGTTCTTTCATGGGGAAGCAGATTTGGGTACCACCCAAGTATTGGCTCACCCATCAACAACCGCTATGTATTTCGTACATTACTGCCAGCCACCATGAATATTGTACAGTACCATAAATACTTGACCACCTGTAGTACATGAAAACCCAACCCACATCAAAACCCCCTCCCCATGCTTACAAGCAAGTACAGCAATCAACCCTCAACTATCACACATCAACTGCAACTCCAAAGCCACCCCTCACCCACTAGGATACCAACAAACCTACCCACCCTTAACAGCACATAGTACATAAAGCCATTTACCGTACATAGCACATTACAGTCAAATCCCTTCTCGTCCCCATGGATGACCCCCCTCAGATAGGGGTCCCTTGAACC
\end{lstlisting}

HV2区测序结果如下:
\begin{lstlisting}
CATTTGGTATTTTCGTCTGGGGGGTGTGCACGCGATAGCATTGCGAGACGCTGGAGCCGGAGCACCCTATGTCGCAGTATCTGTCTTTGATTCCTGCCTCATCCTATTATTTATCGCACCTACGTTCAATATTACAGGCGAACATACTTACTAAAGTGTGTTAATTAATTAATGCTTGTAGGACATAATAATAACAATTGATGTCTGCACAGCCGCTTTCCACACAGACATCATAACAAAAAATTTCCACCAAACCCCCCCCTCCCCCCGCTTCTGGCCACAGCACTTAAACACATCTCTGCCAAACCCCAAAAACAAAGAACCCTAACACCAGCCTAACCAGATTTCAAATTTTATCTTTTGGCGGTATGCCTTTTTGAACAGA
\end{lstlisting}

\section{结果分析}
\subsection{SNP鉴定}
使用Mega12,分别将我的HV1区和HV2去测序结果与班级中所有测序成功的同学的测序结果进行比对,鉴定出SNP位点。HV1区共鉴定出26个SNP位点,详细结果见附表~\ref{tab:hv1_snp};HV2区共鉴定出7个SNP位点,详细结果见附表~\ref{tab:hv2_snp}。
\label{subsec:SNP}

\subsection{进化树构建}
使用MEGA12软件,根据HV1区和HV2区的序列,构建了最大似然法(Maximum Likelihood)进化树。进化树如图~\ref{fig:phylo_tree} 所示。
\begin{figure}[htbp!]
    \centering
    \begin{subfigure}[b]{0.49\textwidth}
        \centering
        \includegraphics[width=\textwidth]{phylo1.png}
        \caption{基于HV1区序列的进化树}
    \end{subfigure}
    \hfill
    \begin{subfigure}[b]{0.49\textwidth}
        \centering
        \includegraphics[width=\textwidth]{phylo2.png}
        \caption{基于HV2区序列的进化树}
    \end{subfigure}
    \caption{基于HV1区和HV2区序列的最大似然法进化树}
    \label{fig:phylo_tree}
\end{figure}
\label{subsec:phylotree}

\subsection{单倍群分析}
为追踪人类进化轨迹,将我的测序结果与从phylotree.org下载的Reconstructed Sapiens Reference Sequence(RSRS)进行比对,共检测到96个SNP位点,详细结果见附表~\ref{tab:rsrs_snp}。

使用Haplogrep 3,初步判断我的mtDNA\textbf{单倍群归属于F} (图~\ref{fig:haplogrep})。根据上一步检测到的SNP位点,在Phylotree上进一步确认,验证Haplogrep 3的预测无误。然而受测序长度的限制,我的mtDNA单倍群无法进一步精确。Haplogrep 3预测结果链接: \\
\href{https://haplogrep.i-med.ac.at/jobs/PA27AU5CVWFQ7ZLlgLWpu8KE1zg6MMUYdtWRP9zHPeFs2DX6In}{https://haplogrep.i-med.ac.at/jobs/PA27AU5CVWFQ7ZLlgLWpu8KE1zg6MMUYdtWRP9zHPeFs2DX6In}

\begin{figure}[htpb!]
    \centering
    \includegraphics[width=0.8\textwidth]{haplogrep.png}
    \caption{Haplogrep 3结果}
    \label{fig:haplogrep}
\end{figure}

\section{实验结论}

本实验通过口腔上皮细胞基因组DNA提取、PCR扩增、测序及生物信息学分析,系统研究了mtDNA高变区(HV1和HV2)的遗传变异特征,得出以下结论:

\subsection{DNA提取与扩增}
\begin{enumerate}
    \item 成功利用Ezup柱式试剂盒从口腔拭子样本中提取基因组DNA,质量符合OD$_{260}$/OD$_{280}$比值1.7--1.9的标准,避免了传统苯酚/氯仿法的毒性风险。
    \item PCR成功扩增mtDNA高变区HV1(约430 bp)和HV2(约400 bp)片段,电泳验证显示目的条带清晰,表明引物设计合理且扩增特异性良好。
    \item HV1区测序质量较高,信号清晰可直接使用;HV2区出现套峰现象,通过正反向测序拼接获得可用序列,说明双向测序策略对提高序列准确性的重要性。
\end{enumerate}

\subsection{群体遗传变异分析}
\begin{enumerate}
    \item 班级群体内HV1区检测到26个SNP位点,HV2区检测到7个SNP位点,反映出mtDNA控制区的高度多态性特征,验证了D-loop区域作为分子标记的有效性。
    \item SNP位点分布不均匀,集中于特定区域,符合mtDNA高变区的突变热点特征,这些变异为群体遗传结构分析和个体识别提供了丰富的遗传标记。
    \item 通过最大似然法构建的系统发育树显示,不同个体间存在明显的分支结构,表明样本间的进化关系和遗传距离差异,HV1和HV2区域的进化树拓扑结构基本一致,相互验证了分析结果的可靠性。
\end{enumerate}

\subsection{进化追踪与单倍群鉴定}
\begin{enumerate}
    \item 将个人测序结果与RSRS(Reconstructed Sapiens Reference Sequence)比对,共检测到96个SNP位点,其中HV2区84个位点(位置146--433),HV1区12个位点(位置16066--16402),这些差异位点记录了现代人类从非洲起源后的进化轨迹。
    \item 利用Haplogrep 3和Phylotree数据库分析,成功鉴定个人mtDNA单倍群归属为\textbf{F单倍群},该单倍群主要分布于东亚和东南亚地区,与中国人群的遗传背景相符,验证了mtDNA单倍群分析在人类起源与迁徙研究中的应用价值。
    \item 受测序长度限制,无法进一步细分至F单倍群的亚型,提示全长mtDNA测序对精确单倍群鉴定的必要性,但现有数据已足以进行宏观的群体归属和地理来源推断。
\end{enumerate}

\subsection{实验技术与方法}
\begin{enumerate}
    \item 掌握了从样本采集、DNA提取、PCR扩增、电泳检测到测序分析的完整实验流程,建立了mtDNA分析的标准操作规范。
    \item 熟练应用MEGA软件进行序列比对、SNP检测和系统发育树构建,运用Haplogrep和Phylotree等在线工具进行单倍群分析,提升了生物信息学数据处理能力。
    \item 理解了mtDNA母系遗传、高突变率、无重组等遗传特性,以及D-loop区域HV1和HV2作为分子标记在群体遗传学、法医学和进化生物学中的应用原理。
\end{enumerate}

综上所述,本实验系统验证了mtDNA高变区在人类进化分析和群体遗传研究中的重要价值,成功通过SNP分析追踪了个体的母系遗传谱系,鉴定出F单倍群归属,为理解东亚人群的遗传结构和迁徙历史提供了分子证据。实验结果表明,mtDNA D-loop区域的序列变异分析是研究人类起源、迁徙和群体分化的有效手段。

\section{思考题}

\subsection*{1. mtDNA相比于核DNA高突变率的原因?}

mtDNA的突变率约为核DNA的10倍,主要原因包括:

\begin{enumerate}
    \item \textbf{氧化应激环境}:线粒体是细胞进行氧化磷酸化的主要场所,在ATP合成过程中会产生大量活性氧自由基(ROS),这些自由基对mtDNA造成持续的氧化损伤,导致碱基修饰和DNA链断裂。由于mtDNA紧邻电子传递链,直接暴露于高浓度ROS环境中,受损伤程度远高于细胞核内的核DNA。
    
    \item \textbf{DNA修复机制不完善}:mtDNA缺乏组蛋白保护,呈裸露状态,更易受到损伤。同时,线粒体内的DNA修复系统相对简单,缺乏核苷酸切除修复(NER)等重要修复途径,虽具备碱基切除修复(BER)和错配修复(MMR)机制,但修复效率远低于核DNA,导致突变更容易积累和固定。
    
    \item \textbf{复制保真度较低}:线粒体DNA聚合酶γ(Pol γ)虽然具有3'→5'外切酶活性进行校对,但其整体保真度仍低于核DNA复制酶复合体。此外,mtDNA复制不依赖细胞周期,持续进行高频率复制,增加了复制错误的累积机会。
    
    \item \textbf{缺乏内含子和非编码区保护}:mtDNA基因组极为紧凑(约16.5 kb),编码序列占比高达93\%,几乎没有内含子和保护性非编码区。这意味着突变更容易直接影响功能基因,虽然有害突变会被选择淘汰,但中性或轻微有害突变更易固定,导致观察到的高突变率。
    
    \item \textbf{母系遗传与瓶颈效应}:mtDNA通过卵细胞母系遗传,在生殖细胞发育过程中经历瓶颈效应,导致突变的随机漂变和快速固定,加速了mtDNA的进化速率。
\end{enumerate}

\subsection*{2. mtDNA多态性有哪些应用?}

mtDNA多态性因其母系遗传、高突变率和无重组等特性,在多个领域具有重要应用:

\begin{enumerate}
    \item \textbf{人类进化与起源研究}:
    \begin{itemize}
        \item 通过mtDNA单倍群分析,追溯现代人类"走出非洲"的迁徙路线和时间,重建人类祖先的地理分布和扩散历史。
        \item 线粒体夏娃理论:利用mtDNA系统发育分析,推测所有现代人类mtDNA均可追溯到约20万年前生活在非洲的共同女性祖先。
        \item 研究不同人群的遗传关系和分化时间,揭示古代人群迁徙、混合和隔离事件。
    \end{itemize}
    
    \item \textbf{法医学与个体识别}:
    \begin{itemize}
        \item 高度降解样本(如骨骼、牙齿、毛发)的个体鉴定,因mtDNA拷贝数高(每个细胞数百至数千份),在核DNA严重降解时仍能成功提取和分析。
        \item 母系亲缘关系鉴定:当无法获取直系父母样本时,可通过比对母系亲属(如外祖母、姨母)的mtDNA确认身份。
        \item 灾难遇难者身份识别和历史人物遗骸鉴定(如末代沙皇家族遗骸鉴定)。
    \end{itemize}
    
    \item \textbf{群体遗传学与生物地理学}:
    \begin{itemize}
        \item 研究种群遗传结构、基因流动和遗传多样性,评估濒危物种的保护遗传学问题。
        \item 分析地理隔离群体的分化程度和迁徙历史,揭示环境因素对种群分布的影响。
        \item 构建生物地理分布模式,推断物种扩散和定居的历史过程。
    \end{itemize}
    
    \item \textbf{疾病遗传学与医学研究}:
    \begin{itemize}
        \item mtDNA突变与线粒体疾病的关联研究:如Leber遗传性视神经病(LHON)与nt11778 G>A突变、MELAS综合征与nt3243 A>G突变等。
        \item 研究mtDNA多态性对复杂疾病(如帕金森病、阿尔茨海默病、糖尿病、癌症)易感性的影响。
        \item 线粒体单倍群与疾病风险、药物代谢、运动能力等表型关联分析。
        \item mtDNA异质性(heteroplasmy)在疾病表达和遗传咨询中的应用。
    \end{itemize}
    
    \item \textbf{古DNA研究与考古学}:
    \begin{itemize}
        \item 从古代样本(如尼安德特人、丹尼索瓦人化石)中提取mtDNA,研究古人类与现代人类的遗传关系。
        \item 追溯农业起源、家养动植物驯化历史和古代贸易路线。
        \item 重建古代文明的人口迁移和文化交流模式。
    \end{itemize}
    
    \item \textbf{物种鉴定与系统发育分析}:
    \begin{itemize}
        \item 利用mtDNA条形码(如COI基因)进行物种快速鉴定,应用于生物多样性调查、食品安全检测和濒危物种保护。
        \item 构建动物系统发育树,解析物种间的进化关系和分类地位。
    \end{itemize}
\end{enumerate}

综上所述,mtDNA多态性已成为遗传学、进化生物学、法医学和医学研究中的重要分子标记,为理解人类起源、疾病机制和生物多样性提供了强有力的工具。

\subsection{3. 分析比较你和同学之间序列的差异。}
详见章节~\ref{subsec:SNP}和~\ref{subsec:phylotree}
\section{收获体会}
学会使用MEGA软件,进行序列对齐,SNP分析,绘制进化树,学会使用phylotree.org查找单倍群,同时了解到Haplogrep 3等快捷查找单倍群的工具。

\section{附加信息}
本次实验的所有原始数据和分析脚本均已上传至GitHub,链接如下: \\
\href{https://github.com/wangblog0/Genetics-experiment-report-10}{https://github.com/wangblog0/Genetics-experiment-report-10}

\newpage
\appendix
\section{附表}

\begin{table}[H]
\centering
\caption{HV1区SNP分析结果}
\label{tab:hv1_snp}
\small
\begin{tabular}{ccccccccccccccccccc}
\toprule
\textbf{位置} & \textbf{1-4} & \textbf{1-5} & \textbf{2-1} & \textbf{2-5} & \textbf{2-6} & \textbf{2-7} & \textbf{3-1} & \textbf{3-3} & \textbf{3-4} & \textbf{3-5} & \textbf{3-6} & \textbf{3-8} & \textbf{4-1} & \textbf{4-3} & \textbf{4-4} & \textbf{4-6} & \textbf{4-7} & \textbf{4-8} \\
\midrule
50 & A & A & A & A & G & A & A & A & A & A & A & A & A & A & A & A & A & A \\
77 & T & T & T & T & T & T & T & T & T & T & T & T & T & T & T & C & T & T \\
95 & C & C & C & T & C & C & C & C & C & C & C & C & C & C & C & C & C & C \\
108 & T & C & T & T & T & T & T & T & T & T & T & T & T & T & T & T & T & T \\
110 & T & T & T & T & T & C & T & T & T & T & T & T & T & T & T & T & T & T \\
113 & G & G & G & A & A & G & A & G & G & G & G & G & G & G & G & G & G & G \\
129 & G & G & G & G & G & G & G & G & G & G & A & G & G & G & G & G & G & G \\
142 & A & A & A & A & A & A & A & A & A & A & A & G & A & A & A & A & A & A \\
146 & A & A & A & A & G & A & A & A & A & A & A & A & A & A & A & A & A & A \\
152 & C & C & C & C & C & C & C & C & C & C & T & C & C & C & C & C & C & C \\
156 & T & T & T & T & C & T & T & T & T & T & T & T & C & T & T & T & T & T \\
163 & C & C & C & C & C & C & C & C & T & C & C & C & C & C & C & C & C & C \\
172 & C & C & C & C & C & C & C & C & C & C & T & C & C & C & C & C & C & C \\
176 & C & C & C & C & C & C & C & C & T & C & C & C & C & C & C & C & C & C \\
207 & T & T & T & T & C & T & T & T & T & T & T & T & T & T & T & T & T & T \\
218 & C & C & C & C & C & C & C & T & C & C & C & C & C & C & C & C & C & C \\
240 & C & T & C & C & C & C & C & C & C & C & C & C & C & C & C & C & C & C \\
241 & C & C & C & A & C & C & C & C & C & C & T & C & C & C & C & C & A & A \\
245 & C & C & C & T & C & C & C & C & C & C & C & C & C & C & C & C & T & T \\
274 & T & T & C & C & C & T & C & C & T & C & C & C & C & C & C & C & C & C \\
288 & T & T & T & T & C & T & T & T & T & T & T & T & T & T & T & T & T & T \\
295 & T & T & T & T & T & T & T & T & T & T & C & T & T & T & T & T & T & T \\
300 & A & A & A & A & A & A & A & G & A & A & A & A & A & A & A & A & A & A \\
303 & A & A & G & G & G & A & G & G & A & G & G & G & G & G & G & G & G & G \\
309 & T & T & T & T & T & T & T & T & T & T & T & T & T & C & T & C & T & T \\
346 & C & C & C & T & T & C & C & C & C & C & T & C & C & C & C & C & T & T \\
\bottomrule
\end{tabular}
\end{table}

\begin{table}[H]
\centering
\caption{HV2区SNP分析结果}
\label{tab:hv2_snp}
\small
\begin{tabular}{ccccccccccccc}
\toprule
\textbf{位置} & \textbf{1-1} & \textbf{1-4} & \textbf{1-5} & \textbf{2-3} & \textbf{2-6} & \textbf{3-1} & \textbf{3-3} & \textbf{3-7} & \textbf{3-8} & \textbf{4-3} & \textbf{4-5} & \textbf{4-8} \\
\midrule
82 & C & C & C & C & C & C & C & C & T & T & C & T \\
84 & T & C & C & T & T & C & T & T & C & T & C & T \\
126 & C & C & C & C & C & C & C & T & C & C & C & C \\
132 & A & A & G & A & A & A & A & A & A & G & A & A \\
167 & A & G & G & A & A & A & A & A & A & A & A & A \\
195 & G & A & G & G & G & T & G & G & G & G & G & G \\
230 & C & C & C & C & C & C & C & C & C & C & T & C \\
\bottomrule
\end{tabular}
\end{table}

\begin{table}[H]
\centering
\caption{基于RSRS参考序列的SNP分析结果}
\label{tab:rsrs_snp}
\small
\begin{minipage}{0.48\textwidth}
\centering
\begin{tabular}{rccc}
\toprule
\textbf{位置} & \textbf{RSRS} & \textbf{HV1} & \textbf{HV2} \\
\midrule
146 & C & - & T \\
152 & C & - & T \\
195 & C & - & T \\
247 & A & - & G \\
310 & T & - & C \\
311 & C & - & T \\
316 & G & - & C \\
318 & T & - & G \\
319 & T & - & C \\
320 & C & - & T \\
322 & G & - & C \\
323 & G & - & T \\
324 & C & - & G \\
325 & C & - & G \\
326 & A & - & C \\
329 & G & - & C \\
330 & C & - & A \\
331 & A & - & G \\
333 & T & - & A \\
334 & T & - & C \\
335 & A & - & T \\
336 & A & - & T \\
338 & C & - & A \\
342 & T & - & C \\
343 & C & - & A \\
347 & G & - & C \\
348 & C & - & T \\
349 & C & - & G \\
350 & A & - & C \\
351 & A & - & C \\
353 & C & - & A \\
354 & C & - & A \\
357 & A & - & C \\
358 & A & - & C \\
362 & C & - & A \\
364 & A & - & C \\
366 & G & - & A \\
368 & A & - & G \\
369 & C & - & A \\
370 & C & - & A \\
372 & T & - & C \\
373 & A & - & C \\
374 & A & - & T \\
375 & C & - & A \\
378 & C & - & A \\
379 & A & - & C \\
380 & G & - & C \\
381 & C & - & A \\
\bottomrule
\end{tabular}
\end{minipage}
\hfill
\begin{minipage}{0.48\textwidth}
\centering
\begin{tabular}{rccc}
\toprule
\textbf{位置} & \textbf{RSRS} & \textbf{HV1} & \textbf{HV2} \\
\midrule
382 & C & - & G \\
383 & T & - & C \\
384 & A & - & C \\
385 & A & - & T \\
386 & C & - & A \\
387 & C & - & A \\
388 & A & - & C \\
389 & G & - & C \\
391 & T & - & G \\
392 & T & - & A \\
394 & C & - & T \\
395 & A & - & T \\
396 & A & - & C \\
398 & T & - & A \\
399 & T & - & A \\
402 & A & - & T \\
404 & C & - & A \\
406 & T & - & C \\
409 & G & - & T \\
410 & G & - & T \\
411 & C & - & G \\
413 & G & - & C \\
414 & T & - & G \\
415 & A & - & G \\
417 & G & - & A \\
418 & C & - & T \\
419 & A & - & G \\
421 & T & - & C \\
425 & A & - & T \\
426 & A & - & T \\
427 & C & - & G \\
429 & G & - & A \\
430 & T & - & C \\
431 & C & - & A \\
432 & A & - & G \\
433 & C & - & A \\
16066 & A & G & - \\
16162 & A & G & - \\
16172 & T & C & - \\
16187 & T & C & - \\
16189 & C & T & - \\
16223 & T & C & - \\
16230 & G & A & - \\
16278 & T & C & - \\
16304 & T & C & - \\
16311 & C & T & - \\
16400 & C & A & - \\
16402 & A & C & - \\
\bottomrule
\end{tabular}
\end{minipage}
\end{table}

\end{document}
